\section{Implementation in Cython}

In order to make the given program faster one part of the monte\_carlo\_ising() function was implemented in Cython.
Because of the already efficient numpy operations it is not necessary to implement the functions compute\_energy() and compute\_magnetization() in C.
The function compute\_act\_error() may be faster in C but as we see later it will not be necessary.\\

The really time intensive part of the program is the trial move.
In this case it is a spin-flip which is done $L^2$ times, with system size $L\times L$, between the num\_sweeps measurements.\\

In order to make this faster the function spin\_flip() is implemented in Cython.
The according C-function is called c\_spin\_flip() and can be seen in code block \ref{spfl}.
The function takes the Temperature in form of beta and the system size L as arguments. 
Furthermore it uses pointer on the energy E, the magnetization mu and the actual state sigma.\\\

\listfile[MyCStyle]{../src/cising_impl.c}{cising\_impl.c}{31}{44}{Spin flip}{spfl}

In fact the function does nothing different from the original python code, but because of C can only use plane arrays the indices will look different.
\begin{align}
	\sigma_{i,j}
		&\rightarrow \sigma_{i+L\cdot j} = \sigma_k
		\label{ijk1}\\
	i
		&=k\%L 
		&&(\text{modulo})
		\label{ijk2}\\
	j
		&=k/L
		&&(\text{integer division})
		\label{ijk3} 
\end{align}

This leads to:
\begin{align}
\sigma_{i\pm 1,j}
	&\rightarrow \sigma_{(i\pm 1)\%L+L\cdot j}
	= \sigma_{(k\pm 1)\%L+k/L\cdot L}
	\label{ijk4}\\
\sigma_{i,j\pm 1}
	&\rightarrow \sigma_{i+\li (j\pm 1)\%L\ri \cdot L}
	= \sigma_{(k\pm L)\%L^2}
	\label{ijk5}
\end{align}

But in C the modulo of negative numbers is defined in a different way than in python and because of this when using the modulo $a\% b$ we have to add $b$ to $a$ to gain $(a+b)\%b$.\\

In order to save disk space no new state sigma is generated but the old one will be overwritten.
Same for E and mu.
They need not to be calculated again but can be simply changed by their change for one spin flip.\\

The function can now replace the inner for-loop in monte\_carlo\_ising().\\

\section{Simulation}

In order to make easier simulations ising.py was extended to take the command line parameter \ls{--L <L1> <L2> ...} which takes all system sizes for which you want to simulate as arguments.
Furthermore the script can now save figures and data.