\documentclass[11pt,a4paper]{scrartcl}

% Pakete
\usepackage[english]{babel}
\usepackage[UKenglish]{isodate}
\usepackage{xcolor}
\usepackage{graphicx}
\usepackage{amsmath}
\usepackage{amssymb}
\usepackage{nicefrac}
\usepackage[utf8]{inputenc}
\usepackage{siunitx}
\sisetup{
output-decimal-marker={.},
exponent-product=\cdot }
\usepackage{esvect}
\usepackage{eqnarray}
\usepackage{placeins}
\usepackage{scrpage2}
\usepackage{nameref}
\usepackage{upgreek}
\usepackage{caption}
\usepackage{subcaption}
\usepackage{bm}
\usepackage{mwe}
\usepackage{tcolorbox}
\usepackage{listings}
\usepackage{xstring}
\usepackage{stringstrings}
\usepackage{floatflt}
\usepackage{pgfplots}
\usepackage{tikz}
%\usepackage{physics}


% Custom colors
\definecolor{deepblue}{rgb}{0,0,0.5}
\definecolor{deepred}{rgb}{0.6,0,0}
\definecolor{deepgreen}{rgb}{0,0.5,0}
\DeclareFixedFont{\ttb}{T1}{txtt}{bx}{n}{12} % for bold
\DeclareFixedFont{\ttm}{T1}{txtt}{m}{n}{12}  % for normal

% tikz
\usetikzlibrary{arrows}

% caption setup
\captionsetup[subfigure]{labelformat=simple, labelsep=colon}
\renewcaptionname{english}{\figurename}{Fig.}
\renewcommand{\thesubfigure}{\arabic{figure}.\arabic{subfigure}}
\renewcommand{\thesubtable}{\arabic{table}.\arabic{subtable}}

% Listning Einstellungen
\lstloadlanguages{python}       % Default highlighting set to "python"
\DeclareCaptionFont{white}{\color{white}}
\DeclareCaptionFormat{listing}{%
  \parbox{0.99\textwidth}{\colorbox{gray}{\parbox{0.99\textwidth}{#1#2#3}}\vskip+5pt}}
\captionsetup[lstlisting]{format=listing, labelfont=white, textfont=white}
\lstset{frame=lrb,xleftmargin=\fboxsep,xrightmargin=-\fboxsep}
\pagestyle{empty}
\lstset{escapeinside={<@}{@>}}
\lstdefinestyle{MyPythonStyle}{
		language=Python,
		numbers=left,
		breaklines=true,
		basicstyle=\ttm,
		otherkeywords={self},             % Add keywords here
		keywordstyle=\ttb\color{deepblue},
		emph={MyClass,__init__},          % Custom highlighting
		emphstyle=\ttb\color{deepred},    % Custom highlighting style
		stringstyle=\color{deepgreen},
		frame=tb,                         % Any extra options here
		showstringspaces=false            % 
		}
\lstdefinestyle{MyCStyle}{
		language=C,
		numbers=left,
		tabsize=4,
		breaklines=true,
		basicstyle=\ttm,
		otherkeywords={self},             % Add keywords here
		keywordstyle=\ttb\color{deepblue},
		emph={MyClass,__init__},          % Custom highlighting
		emphstyle=\ttb\color{deepred},    % Custom highlighting style
		stringstyle=\color{deepgreen},
		frame=tb,                         % Any extra options here
		showstringspaces=false            % 
		}
\renewcommand{\lstlistingname}{Code block}
\renewcommand{\lstlistlistingname}{List of \lstlistingname s}

% Griechische Buchstaben vereinheitlichen
\renewcommand{\alpha}{\upalpha}
\renewcommand{\beta}{\upbeta}
\renewcommand{\gamma}{\upgamma}
\renewcommand{\delta}{\updelta}
\newcommand{\w}{\omega}
\newcommand{\la}{\lambda}

% Eigene mathematische Kommandos
\newcommand{\dd}{\text{d}} 							% Differential
\newcommand{\p}{\partial} 								% Partielles Differential
\newcommand{\D}{\Delta} 								% Fehler / Laplace
\newcommand{\order}[1]{\math\newpage
cal{O}\left( #1 \right)}
\newcommand{\abs}[1]{\left| #1\right|} 					% Betrag
\newcommand{\Max}[1]{\max \left\lbrace #1\right\rbrace} % max{}
\newcommand{\Min}[1]{\min \left\lbrace #1\right\rbrace} % min{}
\newcommand{\diff}[2]{\frac{\text{d} #1}{\text{d} #2}}	% Ableitung
\newcommand{\pdiff}[2]{\frac{\partial #1}{\partial #2}} % Partielle Ableitung
\newcommand{\errprop}[2]{\left| \frac{\partial #1}{\partial #2}\right| \cdot \Delta #2}

% Klammern
\newcommand{\lk}{\left\langle}
\newcommand{\rk}{\right\rangle}
\newcommand{\lb}{\left\lbrace}
\newcommand{\rb}{\right\rbrace}
\newcommand{\lc}{\left(}
\newcommand{\rc}{\right)}
\newcommand{\li}{\left[}
\newcommand{\ri}{\right]}

% Fehlerfortpflanzung
\newcommand{\rel}[1]{\frac{\Delta #1}{#1}}				% Relativer Fehler

% Eigene trigonometrische Funktionen
\newcommand{\Exp}[1]{\text{exp}\left( #1 \right)}		% exp()
\newcommand{\Ln}[1]{\text{ln}\left( #1 \right)}			% ln()
\newcommand{\Log}[1]{\text{log}\left( #1 \right)}		% log()
\newcommand{\Sin}[1]{\text{sin}\left( #1 \right)}     	% sin()
\newcommand{\Cos}[1]{\text{cos}\left( #1 \right)}		% cos()
\newcommand{\Sinz}[1]{\text{sin}^2\left( #1 \right)}	% sin^2()
\newcommand{\Cosz}[1]{\text{cos}^2\left( #1 \right)}	% cos^2()
\newcommand{\Tan}[1]{\text{tan}\left( #1 \right)}		% tan()
\newcommand{\Asin}[1]{\text{asin}\left( #1 \right)}		% asin()
\newcommand{\Acos}[1]{\text{acos}\left( #1 \right)}		% acos()
\newcommand{\Atan}[1]{\text{at\newpage
an} \left( #1 \right)}	% atan()

% Eigene Vektor Kommandos
\newcommand{\tovec}[2]{\begin{pmatrix}#1\\ #2\end{pmatrix}}	% 2D-Vektor
\newcommand{\trvec}[3]{\begin{pmatrix}#1\\ #2\\ #3\end{pmatrix}}	% 3D-Vektor	
\newcommand{\ovec}[1]{\boldsymbol{#1}}

% Eigene Referenz-Kommandos
\newcommand{\eref}[1]{(\ref{#1})}						% Gleichungen
\newcommand{\sref}[2]{\subref{#2}}						% Unterabbildungen
\newcommand{\kref}[1]{\ref{#1} \glqq\nameref{#1}\grqq}  % Kapitel
\newcommand{\lref}[1]{$[#1]$}							% Quellen: \newcommand{\lit}{1} => \lref{\lit}

% listing commandos
\newcommand{\listfile}[7][MyPythonStyle]{
\lstinputlisting[linerange={#4-#5}, firstnumber=#4, caption={#6} \hfill script:  #3, label=#7, style=#1]{#2}}
% arguments:
% \listfile[style]{location/filename}{filename}{firstline}{lastline}{title}{label}
% Imports code from a file 
% You will have to escape the filename and the title
% style is an optional argument.

\newcommand{\ls}[1]{\lstinline@#1@}

% using \lstinline@code@ for code in line
% works with every sign instead of @




% Kopf-/Fusszeile
\pagestyle{scrheadings}
\clearscrheadfoot
\chead{Simulation Methods in Physics I}
\ihead{Worksheet 3}
\ohead{\today}
\ofoot{\pagemark}
\ifoot{Michael Marquardt, Cameron Stewart}

\begin{document}

% Titelseite
\begin{titlepage}

\ \\ \ \\ \ \\

\center\textbf{
\begin{large}
Simulation Methods in Physics I
\end{large} \\ \ \\
\begin{Large}
Worksheet 5: Monte-Carlo
\end{Large}}   \\ \ \\
\ \\ \ \\

\begin{tabular}{lll}
Students: &Michael Marquardt &Cameron Stewart\\ 
matriculation numbers: &3122118 &3216338\\
\end{tabular}

\end{titlepage}

% Inhaltsverzeichnis

% -------------------------------------- Begin Of Document ----------------------------------------

\section{Implementation in Cython}

In order to make the given program faster one part of the monte\_carlo\_ising() function was implemented in Cython.
Because of the already efficient numpy operations it is not necessary to implement the functions compute\_energy() and compute\_magnetization() in C.
The function compute\_act\_error() may be faster in C but as we see later it will not be necessary.\\

The really time intensive part of the program is the trial move.
In this case it is a spin-flip which is done $L^2$ times, with system size $L\times L$, between the num\_sweeps measurements.\\

In order to make this faster the function spin\_flip() is implemented in Cython.
The according C-function is called c\_spin\_flip() and can be seen in code block \ref{spfl}.
The function takes the Temperature in form of beta and the system size L as arguments. 
Furthermore it uses pointer on the energy E, the magnetization mu and the actual state sigma.\\\

\listfile[MyCStyle]{../src/cising_impl.c}{cising\_impl.c}{31}{44}{Spin flip}{spfl}

In fact the function does nothing different from the original python code, but because of C can only use plane arrays the indices will look different.
\begin{align}
	\sigma_{i,j}
		&\rightarrow \sigma_{i+L\cdot j} = \sigma_k
		\label{ijk1}\\
	i
		&=k\%L 
		&&(\text{modulo})
		\label{ijk2}\\
	j
		&=k/L
		&&(\text{integer division})
		\label{ijk3} 
\end{align}

This leads to:
\begin{align}
\sigma_{i\pm 1,j}
	&\rightarrow \sigma_{(i\pm 1)\%L+L\cdot j}
	= \sigma_{(k\pm 1)\%L+k/L\cdot L}
	\label{ijk4}\\
\sigma_{i,j\pm 1}
	&\rightarrow \sigma_{i+\li (j\pm 1)\%L\ri \cdot L}
	= \sigma_{(k\pm L)\%L^2}
	\label{ijk5}
\end{align}

But in C the modulo of negative numbers is defined in a different way than in python and because of this when using the modulo $a\% b$ we have to add $b$ to $a$ to gain $(a+b)\%b$.\\

In order to save disk space no new state sigma is generated but the old one will be overwritten.
Same for E and mu.
They need not to be calculated again but can be simply changed by their change for one spin flip.\\

The function can now replace the inner for-loop in monte\_carlo\_ising().\\

\section{Simulation}

In order to make easier simulations ising.py was extended to take the command line parameter \ls{--L <L1> <L2> ...} which takes all system sizes for which you want to simulate as arguments.
Furthermore the script can now save figures and data.
\section{Binder Parameter}

The Binder parameter U is defined by:
\begin{align}
	U
		&=1-\frac{1}{3}\frac{\lk\mu^4\rk}{\lk\mu^2\rk^2}
		\label{fbinder}
\end{align}

And it is implemented in python just like this:
\listfile{../src/ising.py}{ising.py}{54}{58}{Binder parameter}{cbinder}
\section{Actual Results}

\begin{figure}[ht]
	\centering
	\includegraphics[width=1\textwidth]{../dat/states_L4.pdf}
	\caption{
		Final states for L=4.
		}
	\label{L4}
\end{figure}

\begin{figure}[ht]
	\centering
	\includegraphics[width=1\textwidth]{../dat/states_L16.pdf}
	\caption{
		Final states for L=16.
	}
	\label{L16}
\end{figure}

\begin{figure}[ht]
	\centering
	\includegraphics[width=1\textwidth]{../dat/states_L64.pdf}
	\caption{
		Final states for L=64.
	}
	\label{L64}
\end{figure}

\begin{figure}[ht]
	\centering
	\includegraphics[width=0.9\textwidth]{../dat/E_mu_L_4_16_64.pdf}
	\caption{
		Measurements for the different system sizes L.
	}
	\label{mes}
\end{figure}

When looking on the states in figure \ref{L4} to \ref{L64} you can see, that for lower temperatures there is much more order into the arrangement of spins.
There are big areas which have the same spin all over them.
But what is also the case is that it must not be the whole area of simulation which has the same spin. 
Because of this there are some low temperatures in figure \ref{mes} for which the magnetization has an very low value.

\end{document}

