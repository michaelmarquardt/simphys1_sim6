\section{Actual Results}

\begin{figure}[ht]
	\centering
	\includegraphics[width=1\textwidth]{../dat/states_L4.pdf}
	\caption{
		Final states for L=4.
		}
	\label{L4}
\end{figure}

\begin{figure}[ht]
	\centering
	\includegraphics[width=1\textwidth]{../dat/states_L16.pdf}
	\caption{
		Final states for L=16.
	}
	\label{L16}
\end{figure}

\begin{figure}[ht]
	\centering
	\includegraphics[width=1\textwidth]{../dat/states_L64.pdf}
	\caption{
		Final states for L=64.
	}
	\label{L64}
\end{figure}

\begin{figure}[ht]
	\centering
	\includegraphics[width=0.9\textwidth]{../dat/E_mu_L_4_16_64.pdf}
	\caption{
		Measurements for the different system sizes L.
	}
	\label{mes}
\end{figure}

When looking on the states in figure \ref{L4} to \ref{L64} you can see, that for lower temperatures there is much more order into the arrangement of spins.
There are big areas which have the same spin all over them.
But what is also the case is that it must not be the whole area of simulation which has the same spin. 
Because of this there are some low temperatures in figure \ref{mes} for which the magnetization has an very low value.